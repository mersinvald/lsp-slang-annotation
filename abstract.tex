\begin{abstract}
На сегодняшний день существует много зрелых инструментов и интегрированных сред разработки для широко используемых языков программирования.
Эти инструменты развивались в течении многих лет для удовлетворения большинства потребностей индустрии разработки программного обеспечения.

Однако такие программные продукты зачастую очень сложны и имеют монолитную архитектуру, обычно автономную
от инфраструктуры компилятора языка, что затрудняет замену ключевых компонентов или реализацию поддержки дополнительных языков.

В этой работе мы увидим как строилась архитектура компиляторов и какие осложнения это вызывает в современных реализациях IDE, 
современый подход к построению компиляторов и реализацию гибкой
распределеной интегрированная среды разработки на основе концепции Language Server для мультипарадигменного языка программирования SLang\cite{Zouev2017}.
\end{abstract}