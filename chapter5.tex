\chapter{Evaluation and Discussion}
\label{chap:eval}

Было решено назвать проект Akkadia, по причине модульности данного языкового сервера и возможности реализации поддержки нескольких языков программирования через модульную систему.

Реализация Akkadia согласована со стандартом \textbf{jsonrpc}, используемом в протоколе языкового сервера, поскольку имплементация основана на имплементации языкового сервера Rust.

Протокол языкового сервера включает в себя список методов запроса, которые должны быть обработаны языковым сервером. Akkadia Core содержит реализацию обработки файловых уведомлений в целях обновления ее виртуальной файловой системы и необходимых сервисных методов:
\begin{itemize}
\item ExitNotification
\item Initialized
\item DidOpen
\item DidChange
\item Cancel
\item DidSave
\item DidChangeWatchedFiles
\item ShutdownRequest
\item InitializeRequest
\end{itemize}

Другие методы, такие как уже реализованный \textbf{Completion} будут реализованы как подключаемые модули. Модульная система была реализована таким образом, что она полностью поддерживает модульный подход, описанный в методологии, помимо файлов конфигурации и интеграции модулей в режиме реального времени.

Другие методы языкового сервера ответственны за расширение возможностей анализа кода в редакторе, и будут реализованы в будущем:
\begin{itemize}
\item DidChangeConfiguration
\item GoToDefinition
\item ShowReferences
\item Rename
\item FindImpementations,
\item ShowSymbols
\item Formatting,
\item RangeFormatting,
\end{itemize}
Это может быть реализовано в любом языке, поддерживающем кодирование в JSON.

Также, в продолжение вопроса системы модулей --- она может быть расширена с помощью предоставления модулям большего количества промежуточных данных посредством позволения им использовать структуру состояния State в целях улучшения взаимодействия модулей, зависимых друг от друга посредством наследования. Этим способом Akkadia могла бы поддерживать множество языков, как это было запланировано изначально, предоставляя базовые `Core' языковые модули, ответственные за интеграцию компиляторов, которые зранили бы семантическое представление в дополнительном хранилище, где эта информация будет доступна для использования модулям, наследуемым от корневого языкового модуля.

Принимая во внимание скорость работы системы, текущая `наивная' реализация межпроцессного взаимодействия модулей может быть улучшена использованием общей памяти в тех случаях, где это уместно. Таким образом будет возможно избежать потерь в производительности при передаче крупных структур данных, таких как виртуальная файловая система.

Главное достижение архитектуры и реализации Akkadia --- легкая в использовании \textbf{система модулей}, расширяемая без использования инструментов, библиотек, или ограничений ввода-вывода, привязанных к конкретному языку, что приводит к колоссальной гибкости.