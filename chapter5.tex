\chapter{Evaluation and Discussion}
\label{chap:eval}

The implementation of Akkadia is compliant with \textbf{jsonrpc} standard
used in the Language Server Protocol as it inherits the implementation
from Rust Language Server.

Language Server Protocol includes a list of request methods that Language Server can
or should handle. Akkadia Core has an implementation to handle file notifications
to update its internal Virtual File System and the mandatory service methods:
\begin{itemize}
\item ExitNotification
\item Initialized
\item DidOpen
\item DidChange
\item Cancel
\item DidSave
\item DidChangeWatchedFiles
\item ShutdownRequest
\item InitializeRequest
\end{itemize}

The other methods suck as the implemented \textbf{Completion} and are to be impelemted
as the connectable \textbf{modules}.
Module System was implemented to fully support the modular approach described in the Methodology,
except the configuration files and the on-the-fly module integration.

The other Language Server methods responsible of extending code analysis capabilities of an editor
are to be implemented as a further work, namely
\begin{itemize}
\item DidChangeConfiguration
\item GoToDefinition
\item ShowReferences
\item Rename
\item FindImpementations,
\item ShowSymbols
\item Formatting,
\item RangeFormatting,
\end{itemize}
This can be done in any language that supports JSON encoding.

Also, continuing on module system, it may be extended by providing more auxillary data to the modules,
via allowing them to have an internal storage in the \textbf{State} structure, to improve dependent (via inheritance)
modules interoperation capabilities.
This way Akkadia could support many languages as it was planned initially, via providing the `Core' language modules,
responsible for compiler integration, that would store the Semantic Representation in this additional storage, for the use of the modules
that inherit from the core language module.

Considering the speed of execution, the current `naive' implementation of the module IPC may be improved through
the use of shared memory where it is appropriate. This way there wouldn't be performance losses on transmitting huge data structures
such as the Virtual File System.

The major achievement of the Akkadia architecture and implementation is an easy to use \textbf{Module System},
that is extendable without any Rust or any other language-specific instruments, libraries, or input-output method limitations
hence is providing great flexibility.