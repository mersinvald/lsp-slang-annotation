\chapter{Conclusion}
\label{chap:conclusion}

Языковой сервер --- это новый способ привнести модулярность и повторное использование кода в современные наборы инструментов для интегрированной разработки, который использует компилятор языка для осуществления анализа кода.

С помощью архитектуры, описанной в данной работе, эта модульность может быть развита до следующего уровня, позволяя встраивать любую разновидность анализа, статистические и процие инструменты прямо в языковой сервер, таким образом позволяя расширить простые редакторы кода вплоть до построения исключительно мощных инструментальных наборов. Также данный подход к построению языкового сервера может сделать концепцию ЯС достаточно гибкой, чтобы сделать его легким в использовании инструментов для бизнеса, где разработчики зачастую используют множество специфических внутренних инструментов, которые теперь могут быть встроены в сам языковой сервер.

С позиции разработчика языка программирования, языковой сервер позволяет быстро развернуть готовую к использованию среду различных редакторов кода, поддерживающих иструменты проверки исходного кода для данного языка.

Использование языкового сервера как подхода позволяет сократить время и стоимость разработки инфраструктуры поддержки языка в часто используемых редакторах и IDE как минимум на ресурсы, ныне затрачиваемые на реализацию парсера, поскольку используется парсер, являющийся частью компилятора. Кроме того, одна реализация языкового сервера может поддерживать множество различных редакторов, тогда как плагин обыкновенно способен обеспечить поддержку языка лишь одним редактором или IDE.

В заключение, языковой сервер может оказаться наиболее практичным решением для создания насыщенной инфраструктуры разработки для новых, развивающихся языков, с широкими перспективами для дальнейшего развития через предложенную модульную систему.

