\chapter{Implementation}
\label{chap:impl}

\section{Instruments}
\label{sec:impl:instruments}
To not to get drown undertons of work, it was decided to choose an existing OpenSource
Language server as an implementational basis.

As the researcher favourite language is Rust, a modern programming language that aims
to provide zero-cost abstractions, memory and thread safety and blazing speed,
and provide a decent coding experience at the same
the only choice of relatively mature LS for basis is the RLS (Rust Language Server).

After pruning the rust language m dependend code, RLS can speed up implementation process,
as it is equiped with well-tested basic components:
\begin{itemize}
    \item Language Server Protocol implementation
    \item VFS(Virtual File System)
    \item convenient API to accept and dispatch LS messages
\end{itemize}

\subsection{Autocomplete}
\label{sec:impl:ls_mod:autocomplete}
Description if autocomplete implementation

\subsection{Documentation Generator}
\label{sec:impl:ls_mod:docgen}
Description of documentation generation implementation

\section{Language Server control utility\\ library}
\label{sec:impl:ls_control_api}
Testing and Language Server usage as utility outside from within of client-editor Language Server Protocol client would require
a CLI tool, that would be able to act like and aditor and provide an appropriate interface to run single or batch tasks.