\chapter{Conclusion}
\label{chap:conclusion}

Language Server is a new way to bring modularity and extensive code reuse into
modern integrated development toolsets, that employs language`s compiler to
perform code analysis.

With the architecture described in this thesis, this modularity may be brought to
the next level, allowing to inject virtually any analysis, statistical, or other
tools into the Language Server, making it possible to extend simple code editors
up to a point of building very powerful toolsets. Also this approach to building
a Language Server can make the concept of LS agile enough to make it an
easy to use instrument for enterprise users, who are tending to have a lot of specific
internal tools, which may be now incorporated into the Language Server itself.

From the perspective of a programming language developer, Language Server allows to
rapidly bootstrap a ready-to-use environment of different code editors supporting
the inspection tools for this new language.

The Language Server approach allows to reduce time and cost of development of the language support
in major editors and IDEs at least by the cost of implementing a parser, as it is reused
from the compiler. Additionaly one Language Server implementation can support a variety of
different editors, while a plugin is usually able to provice a language support only for a single editor or IDE.

Concluding, the Language Server may be considered to be the most feasible
solution to make a rich development infrastructure for aspiring new
languages, with a broad path to evolve further through the introduced module system.
